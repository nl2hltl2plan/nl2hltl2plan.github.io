

<!doctype html>
<html>
<head>
  <meta charset="utf-8">
  <meta name="viewport" content="width=device-width initial-scale=1">
  <script src="https://cdn.jsdelivr.net/npm/marked/marked.min.js"></script>
  <link rel="stylesheet" href="https://cdnjs.cloudflare.com/ajax/libs/github-markdown-css/5.1.0/github-markdown-light.min.css">
  <style>
@media (max-width: 767px) { body { padding: 15px; } }
body { box-sizing: border-box;
       min-width: 200px;
       max-width: 980px;
       margin: 0 auto;
       padding: 45px; }
  </style>
  <title>Marked.js</title>
</head>
<body>
  <div id="$m" class="markdown-body"></div>
  <textarea id="$t" style="display:none">
\begin{tcolorbox}[breakable,title=Dialog 2: Generating JSON file]
\textbf{Prompt: }{\it Same as that in Fig.~\ref{fig:Prompt_for_generating_JSON_file}.}
\tcblower
\textbf{GPT-4:}
Alright, let's break down the previously mentioned HTN for loading the dishwasher into JSON format.

First, I will structure the main tasks and the most prominent sub-tasks:

\begin{lstlisting}[language=JSON]
{
   "Task_1": {
      "task_id": "Task_1",
      "task_instruction": "Load dishwasher with items in the specified order",
      "task_related_description": "Place items into the dishwasher",
      "sibling_nodes_condition": "",
      "sub-tasks_of_this_node": ["Task_1.1", "Task_1.2"]
   },
   "Task_1.1": {
      "task_id": "Task_1.1",
      "task_instruction": "Load items into the lower rack in any order",
      "task_related_description": "first load items into the lower rack",
      "sibling_nodes_condition": "",
      "sub-tasks_of_this_node": ["Task_1.1.1", "Task_1.1.2", "Task_1.1.3"]
   },
   "Task_1.2": {
      "task_id": "Task_1.2",
      "task_instruction": "After putting items to the lower rack, load items into the upper rack in the specified sequence",
      "task_related_description": "after that load other things into the upper rack",
      "sibling_nodes_condition": "After loading items into the lower rack",
      "sub-tasks_of_this_node": ["Task_1.2.1", "Task_1.2.2"]
   }
}
\end{lstlisting}

Next, I will expand upon the sub-tasks for the lower and upper racks:

\begin{lstlisting}[language=JSON]
{
   "Task_1.1.1": {
      "task_id": "Task_1.1.1",
      "task_instruction": "Put plates into the lower rack",
      "task_related_description": "put plates into the lower rack",
      "sibling_nodes_condition": "can be done in any sequence with Task_1.1.2 and Task_1.1.3",
      "sub-tasks_of_this_node": []
   },
   The rest is omitted for brevity...
}
\end{lstlisting}
  % "Task_1.1.2": {
   %    "task_id": "Task_1.1.2",
   %    "task_instruction": "Put mugs into the lower rack",
   %    "task_related_description": "put mugs into the lower rack",
   %    "sibling_nodes_condition": "can be done in any sequence with Task_1.1.1 and Task_1.1.3",
   %    "sub-tasks_of_this_node": []
   % },
   % "Task_1.1.3": {
   %    "task_id": "Task_1.1.3",
   %    "task_instruction": "Put utensils into the lower rack",
   %    "task_related_description": "put utensils into the lower rack",
   %    "sibling_nodes_condition": "can be done in any sequence with Task_1.1.1 and Task_1.1.2",
   %    "sub-tasks_of_this_node": []
   % },
   % "Task_1.2.1": {
   %    "task_id": "Task_1.2.1",
   %    "task_instruction": "Put saucers into the upper rack",
   %    "task_related_description": "first put saucers",
   %    "sibling_nodes_condition": "",
   %    "sub-tasks_of_this_node": []
   % },
   % "Task_1.2.2": {
   %    "task_id": "Task_1.2.2",
   %    "task_instruction": "Put cups into the upper rack",
   %    "task_related_description": "and then put cups",
   %    "sibling_nodes_condition": "After putting saucers in the upper rack",
   %    "sub-tasks_of_this_node": []
   % }
This format encapsulates the task structure from the given HTN into the specified JSON format. You can integrate the two code blocks for a complete JSON representation.
\end{tcolorbox}

  </textarea>
  <script>$m.innerHTML = marked.parse($t.value);</script>
</body>
</html>
