\documentclass{article}
\usepackage[most]{tcolorbox}
\usepackage{listings}
\usepackage{subfigure}
\usepackage{float} 
\usepackage{enumitem}
\lstdefinelanguage{json}{
    basicstyle=\normalfont\ttfamily,
    numbers=left,
    numberstyle=\scriptsize,
    stepnumber=1,
    numbersep=8pt,
    showstringspaces=false,
    breaklines=true,
    frame=single,
    backgroundcolor=\color{white},
    stringstyle=\color{blue},
    keywordstyle=\color{red},
    commentstyle=\color{gray},
    morestring=[b]",
    morestring=[d]',
    morecomment=[l]{//},
    morecomment=[s]{/*}{*/},
    morecomment=[l]{\#},
    morekeywords={null,false,true}
}
\begin{document}

\section*{Dialog 1: Generating HTN-like representation}
\textbf{Prompt:}
When given instructions to finish some tasks, human tend to reason in a hierarchical manner. Please decompose the natural language task description into a hierarchical structure based on tasks logical relationships.  Be sure to expand pronouns [e.g., it, them, they] to refer to specific objects, in order to resolve ambiguity. 

\textbf{Example [short description]:}

Task: \textit{"Turn on the stove to heat potato, and chill a sliced pear while the potatoes are heating"}
\begin{enumerate}
    \item  Arrange things
    \begin{enumerate}
        \item Fetch a potato, place the potato on the stove, then turn on the stove [Note: Operations on the same object (potato here) cannot be split into two subtasks]
        \item fetch a pear, sliced the pear, then place the pear in the fridge
    \end{enumerate}
\end{enumerate}

\textbf{Example [short description]:}

Task:\textit{ "Put a towel on the toilet, after putting the towel, get a pot and put it into the fridge."}
\begin{enumerate}
    \item  Arrange things
    \begin{enumerate}
        \item Put a towel on the toilet
        \item Get a pot and put it into the fridge
    \end{enumerate}
\end{enumerate}
    
\textbf{Example [long description]:}

\textit{"First put a black block into the red box, then put a yellow block into the red box. After putting things to red box, you can start to putting toys into the green box (they can be done in any order): put toy bears on the table; dress the doll up before put it into the box. you can put clothes into the blue box in any time, but you need to place all skirts and pants into the blue box before putting in sock into the blue box. eventually, you should pile the red, green, blue box on the table, pile three box can be done in any order",}

\begin{enumerate}
    \item Sort things 
    \begin{enumerate}
    \item  Place objects into different box
    \begin{enumerate}
        \item Place blocks into red boxes 
        \begin{enumerate}
            \item Put the a black block into the red box
            \item Put the a yellow block into the yellow box 
        \end{enumerate}
        \item move toys into green box 
        \begin{enumerate}
            \item Put toy bears into the green box
            \item Dress up the dolls and lays dolls in the green box
        \end{enumerate}
        \item Move clothes into blue box
        \begin{enumerate}
            \item Place skirt into the blue box
            \item Place pant into the blue box
            \item Putt sock into the blue box 
        \end{enumerate}
    \end{enumerate}
    \item  Pile the red, green, blue box
    \begin{enumerate}
        \item Place red box on the left
        \item Place green box in the middle
        \item Move blue box on the right
    \end{enumerate}
    \end{enumerate}
\end{enumerate}

According to the principle:

\textbf{Example} [tasks can be done in any order should be placed under the same parent task]:

task: "At any time do \verb|[task a]|. Mean while, \verb|[task b]| can be done. After \verb|[task a and b]|, do \verb|[task c]|."
\begin{enumerate}
    \item  Arrange things
    \begin{enumerate}
    \item summary of \verb|[task a]| and \verb|[task b]|
    \begin{enumerate}
        \item \verb|[task a]|
        \item \verb|[task b]|
    \end{enumerate}
    \item \verb|[task c]|
    \end{enumerate}
\end{enumerate}
Because \verb|[task a]| and \verb|[task b]| can be done in any order.
    
\textbf{Example} [tasks should be executed in sequential should be placed under the same parent task]:

Task : "do \verb|[task a]| at any time. at any time, do \verb|[task b]|. After \verb|[task a]| finished, do \verb|[task c]|."
\begin{enumerate}
    \item   root task
    \begin{enumerate}
    \item summary of \verb|[task a]| and \verb|[task c]|
    \begin{enumerate}
        \item \verb|[task a]|
        \item \verb|[task c]|
    \end{enumerate}
    \item \verb|[task b]|
    \end{enumerate}
\end{enumerate}
Because \verb|[task a]| and \verb|[task c]| should be done in sequence
    
The requirements for task decomposition are that:
\begin{enumerate}
    \item Ensure that each parent task has at most 3 child tasks, which means you can increase the depth of the hierarchical structure to make it narrower per level.
    \item You do not need to write out the temporal constraints in this step, but when dividing levels, please try to consider tasks relationships.
    \item Directly output the hierarchical task representation without any other texts.
    \item Please keep the root of the task tree as ``1.''
\end{enumerate}

``Place items into the dishwasher. You should put plates, mugs and utensils into the lower rack in any order. After putting items to the lower rack, then put things into upper rack, first put saucers, and then put cups.''

\noindent\rule{16cm}{0.4pt}

\textbf{GPT-4:}
Hierarchical Task Network (HTN) - Load Dishwasher
\begin{enumerate}[label=\arabic*.]
    \item Load dishwasher
    \begin{enumerate}[label=\arabic{enumi}.\arabic*]
        \item Load items into the lower rack
        \begin{enumerate}[label=\arabic{enumi}.\arabic{enumii}.\arabic*]
            \item Put plates into the lower rack (can be done in any sequence with 1.1.2 and 1.1.3)
            \item Put mugs into the lower rack (can be done in any sequence with 1.1.1 and 1.1.3)
            \item Put utensils into the lower rack (can be done in any sequence with 1.1.1 and 1.1.2)
        \end{enumerate}
        \item Load items into the upper rack (should be done after 1.1 is finished)
        \begin{enumerate}[label=\arabic{enumi}.\arabic{enumii}.\arabic*]
            \item Put saucers into the upper rack 
            \item Put cups into the upper rack (should be done after 1.2.1)
        \end{enumerate}
    \end{enumerate}
\end{enumerate}

{This HTN provides a hierarchical breakdown of the given task into its various sub-tasks. The tasks are broken down into sequential or parallel steps, with sequential steps being represented by the use of ``after'' and parallel steps allowing for any sequence.}




\end{document}